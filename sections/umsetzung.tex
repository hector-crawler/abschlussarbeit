\section{Umsetzung}

\subsection{Hardware}

Ausgehend von der theoretischen und praktischen Analyse des bestehenden Roboters (siehe Abschnitt~\ref{sec:konzeption}) wurde ein neues mechanisches Konzept für das Chassis des Krabbelroboters 
entwickelt. Ziel war es, die Komponenten platzsparend, gewichtsoptimiert und modular anzuordnen um die Zugänglichkeit zu erweitern.

Zunächst erfolgte eine Recherche der Maße der Bauteile. Auf Basis dieser Daten wurde eine grobe Anordnung in Form einer Handskizze konzipiert, um ein erstes Gefühl für die Platzverhältnisse zu bekommen.

Das daraus abgeleitete 3D-Modell wurde mithilfe von Autodesk Fusion 360 erstellt. Die Konstruktion wurde in zwei funktionale Ebenen aufgeteilt: 
eine untere Ebene zur zur Steuerung der Motoren und eine obere Ebene zur Steuerung der restlichen Bauteile. 
Die Ebenen wurden als separate Körper modelliert, um spätere Anpassungen zu erleichtern.

Der erste 3D-Druck diente zur Kalibrierung der Maße und Überprüfung der Passgenauigkeit. Dabei traten mehrere Fehler auf: 
Einige Schraubenlöcher waren falsch positioniert, die Inkrementalgeber lagen zu nah beieinander, und die Höhe der oberen Griffe erwies sich als zu gering, 
da die Akkumaße zunächst nur geschätzt worden waren.

Da der alte Roboter bis dahin noch softwareseitig genutzt wurde, konnte der mechanische Umbau erst nach Abschluss der entsprechenden Software-Tests (z.B. Implementierung des Q-Learnings). 
Anschließend wurde das CAD-Modell überarbeitet und an die gewonnenen Erkenntnisse angepasst.

Der zweite Druckvorgang verlief erfolgreich: Die Bauteile passten wie geplant, alle Komponenten konnten montiert werden. 
Der Aufbau des neuen Roboters war damit abgeschlossen und bildete die Grundlage für die weitere softwareseitige Integration.