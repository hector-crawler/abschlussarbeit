\section{Diskussion}

% \textit{zeigen, dass/ob die Konzeption aufgegangen ist; so etwas wie eine Bewertung; Ausblick}

\subsection{Fehlerquellen und methododische Grenzen}

\textbf{Mechanische Fehler}
Im mechanischen Bereich kam es vor allem durch gemachte Fehler im CAD-Design (und die begrenzte Maßgenauigkeit des 3D-Druckverfahrens) zu Problemen. Mehrere Teile mussten nachgedruckt werden, da z.B. Bohrungen oder Bauteilabstände nicht korrekt passten oder Fehler im Design gemacht wurden. 
Außerdem wurde PLA-Filament für die Verbindung von Arm und Hand untereinander und mit der Basis verwendet, was sich negativ auf die Stabilität auswirken könnte. Systematische Tests dazu konnten jedoch leider nicht durchgeführt werden.

\subsection{Bewertung der Umsetzung}

\textbf{Bewertung der mechanischen Umstzung}
Die angestrebte Modularität konnte umgesetzt werden: Alle zentralen Komponenten lassen sich in akzeptabler Zeit austauschen. Das hat sich besonders bei der Arbeit mit der tromversorgung als hilfreich erwiesen, z.B. beim Aufladen des Akkus. Ein Nachteil besteht derzeit noch im Kabelmanagement, das durch die Offenheit und Modularität des Design nicht umgesetzt wurde.

Im Vergleich zum alten Roboter stellt die neue Konstruktion eine deutliche Verbesserung dar. Dort musste beim Abbau bspw. die gesamte obere Ebene zersägt werden, um an einige Kabel zu gelangen.

Das äußere Erscheinungsbild wirkt funktional und aus unserer Sicht auch ansprechend, wobei dies natürlich subjektiv ist. Der Schwerpunkt des Roboters liegt tief, was sich positiv auf die Stabilität auswirken sollte. Systematische Tests konnten aber auch hier leider nicht durchgeführt werden. Eine Verkleinerung der Grundfläche wäre künftig sinnvoll, da aktuell relativ viel Material verbraucht wird, was auch die Druckzeit merklich erhöht. 

Insgesamt konnte das Gewicht des Roboters durch die neue Konstruktion aber reduziert werden. Die genaue Reduktion liegt bei \textbf{xx\,\%} (von \textbf{1{,}32\,kg} auf \textbf{xx\,kg}).

Die optionalen Gestaltungskriterien wurden nicht umgesetzt. 

\subsection{Erkenntnisse und Deutung}

Rückblickend wurde der Aufwand für die mechanische Umsetzung deutlich unterschätzt. Insbesondere die CAD-Konstruktion, das mehrfache Drucken, der Abbau des alten Roboters und der anschließende Aufbau des neuen Systems nahmen mehr Zeit in Anspruch als geplant. Das Zeitmanagement hätte entsprechend besser abgestimmt werden müssen, um Engpässe in der finalen Phase zu vermeiden.

\subsection{Ausblick und offene Fragen}

% \section{Zusammenfassung} % Laut den Vorlagen auf Moodel brauchen wir das nicht?

% \textit{einfach komplett nochmal durch alles durchgehen, dann davon nochmal Fazit ziehen}