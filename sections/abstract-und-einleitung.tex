\section*{Abstract}
\addcontentsline{toc}{section}{Abstract}

\textit{Zusammenfassung (in Englisch); \\ Vier Sätze: 1. worum geht es allgemein, Kontext des Problems; 2. welches Problem wird hier bearbeitet; 3. wie wird das Problem im Wesentlichen gelöst; 4. was ist das hauptsächliche Ergebnis; insg. nicht mehr als 10 Zeilen}

We continued development on a robot that learns to crawl with a two-jointed arm by utilizing various machine learning algorithms. Most of our work was concentrated on improving existing aspects of the project rather than adding new features. These aspects included both the mechanical design of the robot as well as the software running on it. As a result, the whole project was reimplemented from scratch and brought to feature parity with the results of the preceding project.

\section{Einleitung}

Dieses Projekt wurde im Rahmen der Kooperationsphase des Hector-Seminars im Schuljahr 2024/25 durchgeführt. 
Ziel ist es, einen bestehenden einbeinigen Krabbelroboter in softwareseitiger und mechanischer Hinsicht zu überarbeiten und als funktionierendes Demonstrationsobjekt zu realisieren.

Die Aufgabenstellung ist dabei bewusst eher frei gehalten. Der Fokus liegt auf der Entwicklung einer steuerbaren Softwarearchitektur mit ROS und der mechanischen Neukonstruktion des Krabbelsroboters.

Der Rest des Artikels gliedert sich wie folgt:

\begin{itemize}
	\item \textbf{Grundlagen:} Einführung in vorwiegend softwareseitige Konzepte, die später vorausgesetzt werden
	\item \textbf{Konzeption des neuen Krabbelroboters:} Zielsetzungen und Vorgehensweisen, besonders auf mechanischer Seite
	\item \textbf{Realisierung des neuen Krabbelroboters:} Besprechung von unvorhergesehenen Hürden in der Umsetzung
	\item \textbf{Diskussion:} Rückblick und Bewertung der Arbeit, Ausblick für die Zukunft
	\item \textbf{Danksagung, Quellen, Literatur}
\end{itemize}
