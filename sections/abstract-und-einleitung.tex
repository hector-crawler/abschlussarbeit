\section*{Abstract}
\addcontentsline{toc}{section}{Abstract}

\textit{Zusammenfassung (in Englisch); \\ Vier Sätze: 1. worum geht es allgemein, Kontext des Problems; 2. welches Problem wird hier bearbeitet; 3. wie wird das Problem im Wesentlichen gelöst; 4. was ist das hauptsächliche Ergebnis; insg. nicht mehr als 10 Zeilen}

We continued development on a robot that learns to crawl with a two-jointed arm by utilizing various machine learning algorithms. Most of our work was concentrated on improving existing aspects of the project rather than adding new features. These aspects included both the mechanical design of the robot as well as the software running on it. As a result, the whole project was reimplemented from scratch and brought to feature parity with the results of the preceding project.

\section{Einleitung}

Das Projekt wurde im Rahmen der Kooperationsphase des Hector-Seminars im Schuljahr 2024/25 durchgeführt. 
Ziel war es, einen bestehenden einarmigen Krabbelroboter in softwareseitiger und mechanischer Hinsicht zu überarbeiten und als funktionierendes Demonstrationsobjekt zu realisieren.

Die Aufgabenstellung war dabei bewusst eher frei gehalten. Wir haben uns auf die Entwicklung einer steuerbaren Softwarearchitektur mit ROS und die mechanische Neukonstruktion des Krabbelsroboters konzentriert. 
