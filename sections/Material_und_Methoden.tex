\section{Material und Methoden}

\subsection{Mechanischer Aufbau und Konstruktionsprinzipien}

\subsubsection{Analyse des Ausgangsmodells}
\subsubsection{CAD-gestütztes Design und 3D-Druck}

Die Bauteile und das Gehäuse des Roboters wurden mithilfe eines CAD-gestützten Entwicklungsprozesses mit \textit{Autodesk Fusion 360} erstellt. Die digitale Modellierung ermöglichte eine präzise Gestaltung der Bauteile, die direkt überprüft werden konnten. Für den Austausch und die Weiterverarbeitung (insbesondere mit Herrn Prof. Dr. Ihme) wurden die Modelle als standardisierte STEP-Dateien exportiert.

Der anschließende 3D-Druck der Komponenten erlaubte eine schnelle und kostengünstige Fertigung. Durch das additive Herstellungsverfahren wurde nur das tatsächlich benötigte Material eingesetzt, was sowohl Gewicht als auch Ressourcen sparte. Besonders vorteilhaft am Prozess ist zudem, dass keine komplexen Werkzeuge oder Maschinen notwendig sind – Anpassungen ließen sich direkt am Computer vornehmen und unkompliziert umsetzen.

Der kombinierte CAD- und 3D-Druckprozess unterstützt somit eine iterative und flexible Entwicklung, wie sie im Prototypenbau von mobilen Robotern besonders hilfreich ist.

\subsubsection{Gestaltungsschwerpunkte}

Ziel der Weiterentwicklung des übernommenen einarmigen Krabbelroboters war es, ein modular aufgebautes, leichtes und anschaulich gestaltetes System zu schaffen, das sowohl für Demonstrations- als auch Auswertungszwecke geeignet ist. Dabei wurden folgende Anforderungen von uns anfangs festgelegt:

\textbf{Must-Have}
\begin{itemize}
  \item \textbf{Modularität (Basic):} Zentrale Komponenten, insbesondere der Akku, müssen einfach austauschbar und erweiterbar sein, z.\,B. durch Steck- oder Schraubverbindungen.
  \item \textbf{Offenes Design:} Für Demonstrationszwecke müssen alle Komponenten sichtbar und gut zugänglich montiert sein, um einen schnellen Überblick über das System zu ermöglichen.
  \item \textbf{Zentraler, tiefer Schwerpunkt:} Zur Erhöhung der Stabilität und besseren Kraftübertragung beim Antrieb muss ein möglichst zentraler und tiefer Schwerpunkt angestrebt.
  \item \textbf{Leichtbauweise:} Ein geringes Gewicht unterstützt die Energieeffizienz und erleichtert die Steuerung durch kleinere Antriebsmotoren.
  \item \textbf{Mechanische Kompatibilität:} Es muss auf die Verwendung standardisierter Bauteile (z.\,B. M3/M4-Schrauben) geachtet werden, um einfache (De-)Montage und Erweiterbarkeit sicherzustellen.
\end{itemize}

\textbf{Nice-to-Have}
\begin{itemize}
  \item \textbf{Kabelmanagement:} Ordnung und Übersicht im Innenraum durch Führungen oder Halteclips.
%   \item \textbf{Modulare Softwarearchitektur:} Vorzugsweise basierend auf ROS, um flexible Steuerung und Erweiterbarkeit zu ermöglichen.
  \item \textbf{Klicksysteme:} Gehäuseteile, die ohne Werkzeug geöffnet oder getauscht werden können.
%   \item \textbf{Vibrationsdämpfung:} Mechanische Entkopplung der Motoren zur Reduktion von Geräuschen und Verschleiß.
  \item \textbf{Kühlung und Luftzirkulation:} Ein passives oder aktives Kühlsystem angestrebt (Besonders für den Raspberry Pi).
\end{itemize}

\textbf{Optional}
\begin{itemize}
  \item \textbf{Autonomiefunktionen:} Z.\,B. selbstständige Navigation zur Erweiterung des Funktionsumfangs.
  \item \textbf{Erweiterte Sensorik:} Zusätzliche Sensoren wie LiDAR als langfristige Option.
  \item \textbf{Symmetrisches Design:} Eine symmetrische Gestaltung kann die Bewegungsvorhersagbarkeit erhöhen und die Konstruktion vereinfachen.
  \item \textbf{Diagnoseelemente:} LEDs oder Displays zur Anzeige von Systemzuständen (Spannung, Kommunikation, Fehler) neben der bestehenden, digitalen (Fehler-)Meldungen.
\end{itemize}

Diese Zielkriterien bildeten die Grundlage für die Auswahl der Bauteile (bzw. des Akkus) und das mechanische Design des weiterentwickelten Roboters.

\subsubsection{Mechanischer Aufbau und Testanpassungen}

\subsection{Software}

Die Architektur der Crawler-Software, also die Strukturierung des Codes, war einer der wichtigsten Verbesserungspunkte im Vergleich zum Vorgängerprojekt (im Folgenden mit ``Legacy-'' bezeichnet). 

\subsubsection{Robot Operating System (ROS)}

Das Robot Operating System (ROS)\cite{ros} ist ein quelloffenes Softwareframework zur Entwicklung von Robotern. Es bildet eine Middleware zur Verwaltung und einzelner Softwarekomponenten, sowie deren Kommunikation untereinander. Diese Komponenten (genannt ``Nodes'') können jeweils in einer beliebigen Programmiersprache (hier Python) implementiert, und bei Programmstart von ROS ausgeführt werden. Die Nodes laufen in getrennten Prozessen. Kommunikation zwischen diesen Nodes geschieht über Kanäle in ROS (genannt ``Topics''), auf denen Nachrichten von sog. Publisher-Nodes gesendet und von sog. Subscriber-Nodes gelesen werden können. Die Datentypen dieser Nachrichten können pro Topic festgelegt werden. Diese Datentypen können grundlegende Typen wie Zahlen oder Strings, aber auch komplexere, im Code festgelegte Datenstukturen sein.

Im Vergleich zur Legacy-Codebase haben wir das Projekt von ROS 1 zu ROS 2 portiert. ROS 2 ist zum Zeitpunkt der Arbeit die neueste Major-Version von ROS und unterstützt modernere Versionen des Ubuntu-Betriebssystems. Dadurch konnten wir das Betriebssystem auf dem Raspberry Pi von Ubuntu 18.04 auf Ubuntu 24.04 upgraden. Support für ROS 1 endete außerdem im Mai 2025, während der Laufzeit des Projekts, und die Legacy-installation hätte keine Updates mehr erhalten.

\subsubsection{pixi}

pixi ist ein Tool für Softwareentwickler, das die Verwaltung von Software-Bibliotheken vereinfacht und gleichzeitig als Build-System dient. Bibliotheken und Build-Befehle werden von pixi in einer Textdatei im Projektverzeichnis gespeichert, sodass die gesamte Entwicklungsumgebung unabhängig vom Hostsystem reproduzierbar ist.

Um das Programm auf dem Roboter auszuführen, sind mehrere Schritte notwendig:
\begin{itemize}
	\item Der gesamte Quellcode muss auf den Raspberry Pi synchronisiert werden.
	\item Die Web-Komponente wird mithilfe von npm, einem Build-Tool für JavaScript, zu einer statischen HTML-Datei gebaut.
	\item Diese HTML-Datei und der Rest des Programms werden mithilfe von ROS zu einem ausführbaren Programm gebündelt.
	\item Dieses Programm wird mit einem ROS-Befehl gestartet.
\end{itemize}

Da der Build-Prozess der Web-Komponente auf dem Raspberry Pi ziemlich lange dauert, kann die Web-Komponente auch zuerst lokal auf dem eigenen Rechner gebaut werden, damit die resultierende HTML-Datei direkt auf den Pi synchronisiert werden kann.

Mit pixi konnten wir diesen mehrschrittigen Build-Prozess auf dem Roboter stark vereinfachen. Die einzelnen Schritte sind in Pixi-Befehlen wie \texttt{pixi run upload} oder \texttt{pixi run build-web} definiert, was die Arbeit mit der Codebase wesentlich erleichtert. Im Vergleich zum Vorgängerprojekt ist es außerdem nun leichter, den Build-Prozess zu dokumentieren und zu überliefern. Die Befehle zum Starten des Legacy-Programms mussten uns bei der Projektübergabe noch mündlich mitgeteilt werden, da sie in der Legacy-Codebase nicht festgehalten waren.

\subsection{Hardware}

\subsubsection{Auswahl der Hauptkomponenten}

\textbf{Der Microcontroller}

Der Prozessor des Crawlers ist ein Raspberry Pi 4 Model B, der gleiche Single Board Computer (SBC) wie beim Vorgängerprojekt.

Zu Anfang unseres Projekts hatten wir lange Zeit Schwierigkeiten, die neuere Ubuntu-Version auf dem SBC zu installieren. Nach langem Debuggen mit verschiedenen Linux-Distributionen, Netzteilen und SD-Karten kamen wir zu dem Schluss, dass das Modell des Vorgängerprojekts zu Schaden gekommen war, seit wir es übernommen hatten, möglicherweise im Transport.

\textbf{Die Sensoren}

Zur Messung von Drehzahlunterschieden an den Rädern des Krabbelroboters werden zwei Inkrementalgeber eingesetzt. Die Inkrementalgeber des Legacy-Projekt, das Modell \textit{MEC22} der Firma PWB \cite{pwb_me16_datasheet_2011}, werden weiterverwendet. Die Sensoren ermöglichen eine zuverlässige Erfassung von Bewegungen und sind dabei kosteneffizient.

Die gewählte Ausführung bietet eine Auflösung von 500 Zählimpulsen pro Umdrehung (Counts per Revolution, CPR). Damit lässt sich selbst eine sehr feine Wegauflösung realisieren. Bei einem Raddurchmesser von 5\,cm entspricht ein Impuls etwa 0{,}31\,mm Bewegung – ausreichend, um auch kleinste Bewegungen des Roboters zu erfassen.

Die Versorgung erfolgt direkt über den Raspberry Pi, der die notwendige Betriebsspannung zur Verfügung stellt. Die elektrische Verbindung wird über einen 5-poligen Molex-Stecker realisiert.

\textbf{Die Aktoren}

Für die Bewegung des Arms wurden wie im ursprünglichen Projekt Servomotoren des Typs \textit{Dynamixel XL430-W250-T} von Robotis \cite{robotis_xl430} eingesetzt. Die Wiederverwendung dieser Motoren ermöglichte eine einfache Integration in das bestehende System und reduzierte den Entwicklungsaufwand. 

Die Motoren verfügen über ein gutes Verhältnis von Drehmoment (bis zu 1{,}5\,Nm) zu Energieverbrauch und erlauben eine präzise Positionssteuerung. Jeder Motor kann über eine eindeutige ID adressiert werden, was eine Kommunikation über ein gemeinsames Bus-System erlaubt. Zur Datenübertragung wird das TTL-Protokoll verwendet.

Für die Verbindung zwischen den Motoren und dem Steuerrechner kommt der \textit{U2D2-Connector} von Robotis \cite{robotis_u2d2} zum Einsatz. Dieser unterstützt sowohl TTL- als auch RS-485-Verbindungen und ermöglicht die Kommunikation über USB mit dem Steuergerät. Zur stabilen Energieversorgung dient ein \textit{U2D2 Power Hub} \cite{robotis_u2d2_power_hub}, der mehrere Motoren gleichzeitig mit Spannung versorgen kann.


\subsubsection{Spannungsversorgung}

Da der alte Akku, ein \textit{XCell LiPo Cracker CAR} \cite{xcell_akku} mit zwei Zellen, $7,4V$ und $5400mAh$, durch sein hohes Gewicht von $310g$ ca. $25\%$ des Gesamtgewichts des Roboters ausmachte, wurde ein leichterer Akku gesucht.

Der neu ausgewählte \textit{Gens ace Modellbau-Akkupack (LiPo)} \cite{gens_ace_akku} ist mit seinen ebenfalls $7,4V$ Ausgangsspannung ähnlich aufgebaut, wiegt allerdings nur noch $66g$ und ist deutlich kompakter. Aufgrund des geplanten Einsatz als Demonstrationsroboters ist auch die damit verbundenen geringere Laufzeit kein Problem. 

$t_{\text{Versorgung, alt}} = \frac{5400\,\mathrm{mAh}}{1070\,\mathrm{mA}} \approx 5\,\mathrm{h} \newline
\newline
t_{\text{Versorgung, neu}} = \frac{1500\,\mathrm{mAh}}{1070\,\mathrm{mA}} \approx 1{,}4\,\mathrm{h}$

1070mAh sind dabei der maximal gemessene Strom (bei der Bewegung der Motoren) und nicht der Normalzustand.

% \subsubsection{Integration und Inbetriebnahme der Hardware}

\subsection{Methodisches Vorgehen}

\subsubsection{Iterative Entwicklung und Arbeitsteilung}
\subsubsection{Dokumnetation und Versionskontrolle}
