\documentclass[12pt,a4paper]{article}
\usepackage[utf8]{inputenc}
\usepackage[ngerman]{babel}
\usepackage{graphicx}
\usepackage{caption}
\usepackage{geometry}
\usepackage{amsmath}
\usepackage{fancyhdr}
\usepackage{hyperref}
\usepackage[numbers]{natbib}
\usepackage{setspace}
\usepackage{tocloft}

% Seitenränder
\geometry{left=3cm,right=2.5cm,top=2.5cm,bottom=2.5cm}

% Bibliographie-Kommando
\newcommand{\bib}{\bibliographystyle{plainnat}\bibliography{bibliography}}

% Kopf- und Fußzeile
\pagestyle{fancy}
\fancyhf{}
\rhead{Abschlussbericht}
\lhead{Vorname Name}
\rfoot{\thepage}

% Abschnittsabstand
\setlength{\parskip}{1em}
\setlength{\parindent}{0pt}

\begin{document}
	
	% Titelseite
	\begin{titlepage}
		\centering
		\includegraphics[width=0.4\textwidth]{logo.png} \\[1cm] % Logo
		\textbf{\LARGE{Kooperationspartner: Name des Kooperationspartners}} \\[2cm]
		{\Large \textbf{Thema der Arbeit}} \\[2cm]
		
		\vfill
		\textbf{Vorname Nachname} \\
		Straße Hausnummer \\
		PLZ Ort \\
		\vfill
		
		\textbf{Abschlussbericht der Kooperationsphase 2024/25} \\
		Durchgeführt am Institut … \\
		Betreuer: … \\
		\vfill
		
		\today
	\end{titlepage}
	
	% Inhaltsverzeichnis
	\tableofcontents
	\newpage
	
	\section*{Abstract}
	\addcontentsline{toc}{section}{Abstract}
	Kurze Zusammenfassung (in Englisch). \\
	
	\section{Einleitung}
	Kurze Einführung in die Aufgabenstellung.
	
	\subsection{Bibliographie Test}
	
	Die Qualität von Quellcode ist ein zentrales Thema in der Softwareentwicklung. Meyer beschreibt in seinem Buch klare Prinzipien für sauberen Code \cite{meyer2008clean}. 
	
	Das HTTP-Protokoll und seine Bedeutung für Webanwendungen ist in den entsprechenden RFCs dokumentiert \cite{rfc7231}.
	
	Ein modernes JavaScript-Framework, das diese Prinzipien unterstützt, ist Vue.js, dessen Entwicklung offen auf GitHub verfolgt werden kann \cite{github_vue}.
	
	\section{Material und Methoden}
	Welche Materialien und Methoden wurden eingesetzt? Ggf. auch Vorversuche.
	
	\section{Ergebnisse}
	Darstellung der gewonnenen Daten, ggf. mit statistischer Auswertung.
	
	\section{Diskussion}
	Diskussion möglicher Fehlerquellen, Interpretation der Ergebnisse, Ausblick.
	
	\section{Fehlerdiskussion, Zusammenfassung und Ausblick}
	Zusätzliche Reflexion, offene Fragen und mögliche weitere Untersuchungen.
	
	\section*{Danksagung}
	\addcontentsline{toc}{section}{Danksagung}
	Dank an Betreuer, Förderer etc.
	
	\section*{Quellen}
	\addcontentsline{toc}{section}{Quellen}
	\bib
	
	\appendix
	\section{Anhang}
	Weitere Bilder, Daten, Abkürzungen usw.
	
	\section*{Selbstständigkeitserklärung}
	\addcontentsline{toc}{section}{Selbstständigkeitserklärung}
	Hiermit versichere ich, dass ich diese Arbeit unter der Beratung durch … selbstständig verfasst habe und keine anderen als die angegebenen Quellen und Hilfsmittel benutzt wurden, sowie Zitate kenntlich gemacht habe. \\[2cm]
	
	Ort, Datum \hfill \rule{5cm}{0.4pt} \\
	\hfill (Unterschrift)
	
\end{document}
